\documentclass{article}
\usepackage[utf8]{inputenc}
\usepackage{amsmath}
\usepackage{amsfonts}
\usepackage{amssymb}
\usepackage{graphicx}
\usepackage{hyperref}
\usepackage{listings}
\usepackage{xcolor}

\title{Pandoc LaTeX Example Document}
\author{Example Author}
\date{\today}

\lstset{
    basicstyle=\ttfamily\small,
    keywordstyle=\color{blue},
    stringstyle=\color{red},
    commentstyle=\color{green},
    numbers=left,
    numberstyle=\tiny,
    frame=single,
    breaklines=true
}

\begin{document}

\maketitle

\tableofcontents
\newpage

\section{Introduction}

This is an example LaTeX document demonstrating various features that can be rendered using pandoc with LaTeX. This document showcases mathematical equations, code listings, tables, and other common document elements.

\section{Mathematical Expressions}

\subsection{Inline Mathematics}
Here's an inline equation: $E = mc^2$ and another one: $\sum_{i=1}^{n} x_i$.

\subsection{Display Mathematics}
Here's a display equation:
\begin{equation}
\int_{-\infty}^{\infty} e^{-x^2} dx = \sqrt{\pi}
\end{equation}

Matrix example:
\begin{equation}
A = \begin{pmatrix}
a_{11} & a_{12} & \cdots & a_{1n} \\
a_{21} & a_{22} & \cdots & a_{2n} \\
\vdots & \vdots & \ddots & \vdots \\
a_{m1} & a_{m2} & \cdots & a_{mn}
\end{pmatrix}
\end{equation}

\section{Code Examples}

\subsection{Python Code}
\begin{lstlisting}[language=Python, caption=Python example]
def fibonacci(n):
    """Generate Fibonacci sequence up to n terms."""
    if n <= 0:
        return []
    elif n == 1:
        return [0]
    elif n == 2:
        return [0, 1]

    sequence = [0, 1]
    for i in range(2, n):
        sequence.append(sequence[i-1] + sequence[i-2])

    return sequence

# Example usage
print(fibonacci(10))
\end{lstlisting}

\subsection{JavaScript Code}
\begin{lstlisting}[language=JavaScript, caption=JavaScript example]
function quickSort(arr) {
    if (arr.length <= 1) {
        return arr;
    }

    const pivot = arr[Math.floor(arr.length / 2)];
    const left = arr.filter(x => x < pivot);
    const middle = arr.filter(x => x === pivot);
    const right = arr.filter(x => x > pivot);

    return [...quickSort(left), ...middle, ...quickSort(right)];
}

console.log(quickSort([3, 6, 8, 10, 1, 2, 1]));
\end{lstlisting}

\section{Tables}

\begin{table}[h]
\centering
\caption{Sample Data Table}
\begin{tabular}{|l|c|r|}
\hline
\textbf{Name} & \textbf{Age} & \textbf{Score} \\
\hline
Alice & 25 & 95.5 \\
Bob & 30 & 87.2 \\
Charlie & 22 & 92.8 \\
Diana & 28 & 89.1 \\
\hline
\end{tabular}
\end{table}

\section{Lists}

\subsection{Itemized List}
\begin{itemize}
    \item First item
    \item Second item with \textbf{bold text}
    \item Third item with \textit{italic text}
    \item Fourth item with \texttt{monospace text}
\end{itemize}

\subsection{Enumerated List}
\begin{enumerate}
    \item First numbered item
    \item Second numbered item
        \begin{enumerate}
            \item Nested item A
            \item Nested item B
        \end{enumerate}
    \item Third numbered item
\end{enumerate}

\section{Text Formatting}

This paragraph demonstrates various text formatting options:
\begin{itemize}
    \item \textbf{Bold text}
    \item \textit{Italic text}
    \item \texttt{Monospace/typewriter text}
    \item \underline{Underlined text}
    \item \textsc{Small caps text}
    \item \textsuperscript{Superscript} and \textsubscript{subscript}
\end{itemize}

\section{Special Characters and Symbols}

Mathematical symbols: $\alpha, \beta, \gamma, \delta, \epsilon, \pi, \sigma, \omega$

Special characters: \&, \$, \%, \#, \_, \{, \}

Quotation marks: ``double quotes'' and `single quotes'

Em-dash: --- and en-dash: --

\section{Cross-References}

This document has multiple sections. For example, see Section~\ref{sec:conclusion} for the conclusion.

You can also reference equations like Equation~\eqref{eq:integral} above.

\section{Conclusion}
\label{sec:conclusion}

This example document demonstrates various LaTeX features that can be processed by pandoc. It includes mathematical equations, code listings, tables, lists, text formatting, and cross-references. This should provide a comprehensive test case for pandoc/LaTeX rendering capabilities.

\end{document}